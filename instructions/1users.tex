\documentclass[twoside,a4paper,12pt]{article}

\usepackage[utf8x]{inputenc}
\usepackage[T1]{fontenc}
\usepackage[colorlinks,
  citecolor=black,linkcolor=black,urlcolor=black]{hyperref}
\usepackage{enumitem}
\usepackage{url}
\usepackage{listings}
\usepackage{pstricks}
\usepackage{pgfplots}
\usepackage{listings}
\usepackage{xcolor}
\usepackage[tikz]{bclogo}
\definecolor{dkgreen}{rgb}{0,0.6,0}
\definecolor{gray}{rgb}{0.5,0.5,0.5}
\definecolor{mauve}{rgb}{0.58,0,0.82}

\lstset{frame=tb,
  language=PHP,
  aboveskip=3mm,
  belowskip=3mm,
  showstringspaces=false,
  columns=flexible,
  basicstyle={\footnotesize\ttfamily},
  numbers=none,
  numberstyle=\tiny\color{gray},
  keywordstyle=\color{blue},
  commentstyle=\color{dkgreen},
  stringstyle=\color{mauve},
  breaklines=true,
  breakatwhitespace=true,
  tabsize=3
}

\usepackage{amsmath}
\usepackage{amssymb}
\usepackage{amsthm}

\usepackage{natbib} % bibtex

\usepackage{multicol}
\usepackage[hmargin={.12\paperwidth,.18\paperwidth},
  vmargin=.18\paperwidth,headheight=15pt]{geometry}

% Entêtes et pieds de page
\usepackage{fancyhdr}
% Configuration des en-têtes et pieds-de-page : tiré du User Guide
\fancyhead{} % clear all header fields
\fancyhead[RO,LE]{\bfseries Polytech Tours DI 3A}
\fancyhead[LO,RE]{\bfseries DB Practical Work}
\fancyfoot{} % clear all footer fields
\fancyfoot[RO,LE]{\thepage}
% Par défaut, on utilise le style fancy
\pagestyle{fancy}
% Pour la page de garde, on redéfinit le style plain
\fancypagestyle{plain}{%
  \fancyhf{} % clear all header and footer fields
  \fancyfoot[RO,LE]{\thepage}
  \renewcommand{\headrulewidth}{0pt}
  \renewcommand{\footrulewidth}{0pt}}

\usepackage[english]{babel}

\newenvironment{foreignpar}[1][english]{%
    \em\selectlanguage{#1}%
}{}
\newcommand*{\foreign}[2][english]{%
    \emph{\foreignlanguage{#1}{#2}}%
}

\title{DB Practical Work 1:\\The User model}

\date{}

\begin{document}

\maketitle

%% RESUME -----------------------------------------------------------------
\begin{abstract}
  The following subject aims at implementing the data handling for users in a twitter-like web-application. Implementations are to be done in the file \texttt{model\_student/user.php}
\end{abstract}

\tableofcontents

\clearpage

\section{Requirement}
To fulfill this work, you will need the following elements:

\begin{itemize}
\item A working environment with db connection to both app and test databases (see \texttt{0setup.pdf}).
\item At least the tables modeling \textbf{user} and \textbf{followings} in \texttt{sql/schemas.sql}.
\end{itemize}

\section{Work to do}
You have to fill out the functions defined in the file \texttt{model\_student/user.php}

These functions are used in the application to get access to the database. Therefore, these functions must observe some rules about both input data (the formats of the parameters of the functions) and output data (the returned values).

In the functions, you can access to the PDO object by using the following instruction:

\begin{lstlisting}
$db = \Db::dbc();
\end{lstlisting}

Then, you can perform queries using \texttt{\$db} like a PDO object:
\begin{lstlisting}
$db = \Db::dbc();
$result = $db->query('SELECT * FROM user');
\end{lstlisting}

When you completed all the functions, you can check them by using the available unit tests.

\section{The User entity}

\subsection{Presentation}
The User entity represents a user and its properties:

\begin{itemize}
\item its login username (used in URLs and during identification)
\item its displayed name (which is a name to be displayed in the application
\item its hashed password (for identification)
\item its email
\item its avatar (or profile picture)
\end{itemize}

\subsection{Creation, getting, password management and authentication}
The following functions have to be coded before running the unit tests. They are in charge of handling both the creation and the fetching of the user objects.

\subsubsection{\texttt{hash\_password(\$password)}}
This function takes a clear password as a parameter and returns a hashed version of it.

\subsubsection{\texttt{create(\$username, \$name, \$password, \$email, \$avatar\_path)}}

This function inserts a user in database. \textit{It is to be noted that the password must be hashed (using \texttt{hash\_password(\$password)})}

The function returns id of the newly inserted user.

It doesn't check whether the username is already taken or not.

\subsubsection{\texttt{get(\$id)}}
This function gets a user with a given id (the one given in parameter). Returns \texttt{null} if no user with the given id were found.

The application asks for a particular output: \texttt{get(\$id)} must return a stdClass PHP object. Such an object can be declared as follows:

\begin{lstlisting}
$o = (object) array(
	"attribute" => "value"
);
\end{lstlisting}

In the case of our User entity, an object will be owning the following attributes:

\begin{lstlisting}
$o = (object) array(
    "id" => 1337,
    "username" => "yrlgtm",
    "name" => "User 1",
    "password" => "hashed",
    "email" => "yrlgtm@gmail.com",
    "avatar" => "images/sddfvjdfvj.png" 
);
\end{lstlisting}

\subsubsection{\texttt{get\_by\_username(\$username)}}
This function returns a user matching the given username (same return format as in \texttt{get(\$id)}). Returns \texttt{null} if no user were found.

\subsubsection{\texttt{check\_auth(\$username, \$password)}}
Tries to authenticate a username with a given password. Returns the user object (same return format as in \texttt{get(\$id)}) if everything went fine. Returns \texttt{null} else. \textit{This function \textbf{does} need to hash the password}

\subsubsection{\texttt{check\_auth\_id(\$id, \$password)}}
Tries to authenticate a user id with a given password. Returns the user object (same return format as in \texttt{get(\$id)}) if everything went fine. Returns \texttt{null} else. \textit{This function \textbf{doesn't} need to hash the password}, because \texttt{\$password} is already given in its hashed form.

\subsection{User modification and deletion}
After the user objects being created, its attributes can be modified. The following functions are in charge of this and should be coded before running the unit tests.

\subsubsection{\texttt{modify(\$uid, \$username, \$name, \$email)}}
This function updates a user whose id is \texttt{\$uid}. It doesn't check whether the new username is already taken or not.


\subsubsection{\texttt{change\_password(\$uid, \$new\_password)}}
This function updates only a user's password. This function must hash the new password.

\subsubsection{\texttt{change\_avatar(\$uid, \$avatar\_path)}}
This function changes the avatar of the user.

\subsubsection{\texttt{destroy(\$id)}}
This function deletes a user entry.

\begin{bclogo}[logo=\bcattention, noborder=true, barre=none]{Important!}
	The deletion behaviour must be allowing deleting user which is following a user or followed by a user. \textbf{To see and modify the deletion policy in phpMyAdmin, you must go in the table view → Structure → Relation view.}
		
\end{bclogo}

\subsection{Searching and listing}
The user objects can be searched for or listed. The two following functions can be coded separately.

\subsubsection{\texttt{search(\$string)}}
This function searches for users by query on both username and displayed name.

\subsubsection{\texttt{list\_all()}}
This function returns an array of every users objects (same return format as in \texttt{get(\$id)}).

\section{Following users}

\subsection{Presentation}
Users can follow each others. Following someone enables a user to receive in their timeline the other user's posts.

\begin{itemize}
\item A user's followers are the users following him/her
\item A user's following are the users he/she follows
\end{itemize}

Every functions relative to following must all be coded before running the unit tests. 

\subsection{\texttt{follow(\$id, \$id\_to\_follow)}}
This function creates a "follow" association between two users.

\subsection{\texttt{unfollow(\$id, \$id\_to\_unfollow)}}
This function deletes a "follow" association between two users.

\subsection{\texttt{get\_followers(\$uid)}}
This function returns an array of objects for every users that follow a given user.

\subsection{\texttt{get\_followings(\$uid)}}
This function returns an array of objects for every users that a given user follows.

\end{document}
