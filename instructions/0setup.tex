\documentclass[twoside,a4paper,12pt]{article}

\usepackage[utf8]{inputenc}
\usepackage[T1]{fontenc}
\usepackage[colorlinks,
  citecolor=black,linkcolor=black,urlcolor=black]{hyperref}
\usepackage{enumitem}
\usepackage{url}
\usepackage{listings}
\usepackage{pstricks}
\usepackage{pgfplots}
\usepackage{listings}
\usepackage{hyperref}
\definecolor{dkgreen}{rgb}{0,0.6,0}
\definecolor{gray}{rgb}{0.5,0.5,0.5}
\definecolor{mauve}{rgb}{0.58,0,0.82}

\lstset{frame=tb,
  language=PHP,
  aboveskip=3mm,
  belowskip=3mm,
  showstringspaces=false,
  columns=flexible,
  basicstyle={\footnotesize\ttfamily},
  numbers=none,
  numberstyle=\tiny\color{gray},
  keywordstyle=\color{blue},
  commentstyle=\color{dkgreen},
  stringstyle=\color{mauve},
  breaklines=true,
  breakatwhitespace=true,
  tabsize=3
}

\usepackage{amsmath}
\usepackage{amssymb}
\usepackage{amsthm}

\usepackage{natbib} % bibtex

\usepackage{multicol}
\usepackage[hmargin={.12\paperwidth,.18\paperwidth},
  vmargin=.18\paperwidth,headheight=15pt]{geometry}

% Entêtes et pieds de page
\usepackage{fancyhdr}
% Configuration des en-têtes et pieds-de-page : tiré du User Guide
\fancyhead{} % clear all header fields
\fancyhead[RO,LE]{\bfseries Polytech Tours DI 3A}
\fancyhead[LO,RE]{\bfseries DB Practical Work}
\fancyfoot{} % clear all footer fields
\fancyfoot[RO,LE]{\thepage}
% Par défaut, on utilise le style fancy
\pagestyle{fancy}
% Pour la page de garde, on redéfinit le style plain
\fancypagestyle{plain}{%
  \fancyhf{} % clear all header and footer fields
  \fancyfoot[RO,LE]{\thepage}
  \renewcommand{\headrulewidth}{0pt}
  \renewcommand{\footrulewidth}{0pt}}

\usepackage[english]{babel}

\newenvironment{foreignpar}[1][english]{%
    \em\selectlanguage{#1}%
}{}
\newcommand*{\foreign}[2][english]{%
    \emph{\foreignlanguage{#1}{#2}}%
}

\title{DB Practical Work 0:\\Setting the system up}

\date{\today}

\begin{document}

\maketitle

%% RESUME -----------------------------------------------------------------
\begin{abstract}
  The following leaflet gives the steps to set the development environment up.
\end{abstract}

\tableofcontents

\clearpage

\section{Work to do}
You have to set your system up in order to get a proper development environment.

You must install the environment by following section \ref{sec:installation}.

\section{Installation}
\label{sec:installation}

Using the virtual machine is recommended but you can also use a personnal machine to run the Vagrant image. \textbf{Choose one of the following subsection}.

\newpage
\subsection{On the Virtual Machine}
\begin{enumerate}
\item Get the VM called "VMWARE UBUNTU 16 64 LTS - DBPROJECT" in \texttt{D:\textbackslash VM Productions\textbackslash}
\item Start it
\item Go in \texttt{/home/ubuntu/db-project/} : here are the sources already installed.
\item Put:
\begin{itemize}
	\item the SQL table creation commands (\texttt{CREATE TABLE}) inside\\ \texttt{/home/ubuntu/db-project/sql/schemas.sql}
	\item the SQL entries creation commands (\texttt{INSERT INTO})\\ inside \texttt{/home/ubuntu/db-project/sql/entries.sql}
\end{itemize}
\item Go in \texttt{/home/ubuntu/db-project/scripts} and click on \texttt{populate\_db.sh}
\end{enumerate}

If you open a web browser inside the VM and go to \url{http://127.0.0.1}, you will find the web application. PHPMyAdmin is accessible at \url{http://127.0.0.1/phpmyadmin/} (user: \texttt{root}, password: \texttt{password}).

There are many scripts to manipulate the VM : they are in\\ \texttt{/home/ubuntu/db-project/scripts}. You can click on them to execute them.

Their roles are described in section \ref{sec:scripts}. 

\newpage
\subsection{Using the lightweight Vagrant image}
You will need the following elements for the software to work:
\begin{enumerate}
\item Virtualbox (\url{https://www.virtualbox.org/})
\item Vagrant (\url{https://www.vagrantup.com/})
\end{enumerate}

Once you have got all the requirements fulfilled, you can proceed with the following steps :

\begin{enumerate}
\item Download \url{https://github.com/prafiny/db-project/archive/master.zip}
\item Unzip the archive somewhere.
\item Put:
\begin{itemize}
	\item the SQL table creation commands (\texttt{CREATE TABLE}) inside \texttt{sql/schemas.sql}
	\item the SQL entries creation commands (\texttt{INSERT INTO})\\ inside \texttt{sql/entries.sql}
\end{itemize}
\item Then
\begin{itemize}
	\item For Windows: go in the folder \texttt{scripts/win}, click on the \texttt{launch\_vagrant} script.
	\item For Linux/MacOS: go in the folder \texttt{scripts/} and execute the \texttt{launch\_vagrant} script.
\end{itemize}
\item Wait
\end{enumerate}
The Vagrant image should be ready and running. If you open a web browser and go to \url{http://127.0.0.1:8080}, you will find the web application. PHPMyAdmin is accessible at \url{http://127.0.0.1:8080/phpmyadmin/} (user: \texttt{root}, password: \texttt{password}). To shut the image down, you can use the script \texttt{stop\_vagrant}.

There are many scripts to manipulate the image :
\begin{itemize}
	\item For Windows in \texttt{scripts/win}
	\item For Linux/MacOS in \texttt{scripts}
\end{itemize}

Their roles are described in section \ref{sec:scripts}. 
\newpage
\section{Scripts}
\label{sec:scripts}

\begin{description}
	\item[\texttt{populate\_db}] This script creates the tables and entries using the files in the folder \texttt{sql/}
	\item[\texttt{snapshot\_db}] This script saves the current \texttt{dbproject\_app} mysql database into the \texttt{sql/} files
	\item[\texttt{tests}] Launches the unit tests.
	\item[\texttt{update}] Update the source code and dependencies.
\end{description}
\end{document}
